\documentclass[a4paper]{article}
\usepackage[T1]{fontenc}
\usepackage{enumitem}
\usepackage[utf8]{inputenc}
\usepackage[italian]{babel}
%\usepackage{geometry} 
%\geometry{a4paper,top=3cm,bottom=3cm,left=3.5cm,right=3.5cm,% heightrounded,bindingoffset=5mm}

\begin{document}

\author{Manuel Trivilino, Davide Salaorni, Luca Terracciano}

\title{\Large Data4Help\\
\Large RASD Requirements Analysis and Specification Document
}

\maketitle
\newpage

\tableofcontents
\newpage

\section{Introduction}

\subsection{Purpose}

\subsection{Scope}

\paragraph{Goals}

\begin{enumerate}[label={[G.\arabic*]}]
	
	%Data4Help
	
	\item Data are collected only for people who are registred to Data4Help.
	\item Individual data can be accessed only if the user agreed to share his/her data with a third party service.
	\item Data4Help allows to register two kind of users: single person and third parties.
	\item Third parties can request for accessing a specific person data.
	\item A person can allow or deny the access to their personal data from a specific third party service that requests for them.
	\item Third parties can request for anonymazed data using various filters.
	\item Data4Help can decide to not share anonymazed data with the third party service if the set of filters applied create a subset of the users smaller than 1000.
	\item Once a third party service has obtained the access to the data it can subscribe for new data.
	
	%AutomatedSOS
	\item AutomatedSOS monitors the health status of people registred to it.
	\item AutomatedSOS sends a notification to an ambulance if the user health parameter are below a certain tresholds. 
	
	%Track4Run
	\item Track4Run allows users to create a run and define a path for it, the participants and the spectators.
	\item The spectators can see the participants position during the run.
	\item Users can accept or refuse to participate in a run.
	
\end{enumerate}

\subsection{Definitions, Acronyms, Abbreviation}

\subsection{Revision History}

\subsection{Reference Documents}

\subsection{Document Structure}

\section{Overall Description}

\subsection{Product Perspective}

\subsection{Product Functions}

\subsection{User Characteristics}

\subsection{Assumption, Dependencies and Constraints}

\subsubsection{Domain Assumptions}


\begin{itemize}[label={[D.\arabic*]}]
    \item Users by registering to TrackMe give the consense to share anonymous data with third parties.
    \item Registered people log in with his own account, collected data belongs to the account owner and the personal informations (Age, address, sex...) given from the users are correct.
    \item Users log in with devices from which is possible to collect information and data (like smartphones, smartwatches, laptops with an internet connection, sensors like GPS, accelerometer, pulse sensor...).
    \item The data collected from the devices is reliable and accurate.
\end{itemize}


\section{Specific Requirements}

\subsection{External Interface Requirements}

\subsubsection{User Interfaces}

\subsubsection{Hardware Interfaces}

\subsubsection{Software Interfaces}

\subsubsection{Communication Interfaces}

\subsection{Functional Requirements}

\subsection{Performance Requirements}

\subsection{Design Constraints}

\subsubsection{Standards Compliance}

\subsubsection{Hardware Limitations}

\subsubsection{Any other Constraint}

\subsection{Software System Attributes}

\subsubsection{Reliability}

\subsubsection{Availability}

\subsubsection{Security}

\subsubsection{Maintainability}

\subsubsection{Portability}

\section{Formal Analysis Using Alloy}

\section{Effort Spent}

\section{References}


\end{document}