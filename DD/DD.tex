\documentclass[a4paper]{article}
\usepackage[T1]{fontenc}
\usepackage[utf8]{inputenc}
\usepackage[italian]{babel}
\usepackage{courier}
\usepackage{float}
\usepackage{subcaption}
\usepackage{enumitem}
\usepackage{graphicx}
\usepackage{rotating}
\usepackage{tikz}
\usepackage{makecell}
\graphicspath{{./images/}}
\usepackage{tabularx}
\usepackage{hyperref}
%\usepackage{geometry} 
%\geometry{a4paper,top=3cm,bottom=3cm,left=3.5cm,right=3.5cm,heightrounded,bindingoffset=5mm}

\begin{document}

\author{Manuel Trivilino, Davide Salaorni, Luca Terracciano}

\title{\Large \textbf{Data4Help DD}\\
\textbf{Design Document}
    \begin{figure}[h]
        \centering
        \includegraphics[width=270pt]{data4helpblu}
        \label{fig:my_label_1}
    \end{figure}
    \vspace{-2.4cm}
    \begin{figure}[h]
        \centering
        \includegraphics[width=200pt]{AutomatedSOSlogo.png}
        \label{fig:my_label_2}
    \end{figure}
    \vspace{-2cm}
    \begin{figure}[h]
        \centering
        \includegraphics[width=150pt]{Track4Runlogo.png}
        \label{fig:my_label_3}
\end{figure}
}

\maketitle
\newpage

\tableofcontents
\newpage

\section{Introduction}

\subsection{Purpose}
This document aims to give to the developer team more technical details about the implementation of Data4Help, AutomatedSOS and Track4Run identifying the high level architecture, the design patterns used and the plans that will guide the development of the code.
\vspace{.5 cm}

\subsection{Scope}
\textbf{TrackMe} is a company that develops software-based services for third parties and for consumers. The main service is called Data4Help. \\
\textbf{Data4Help} is a service that allows third parties to monitor the location and the health status of individual, it handles a policy of permissions for each user and collects individual’s data from their personal devices.
The service supports the registration of individuals who, by registering, agree that TrackMe acquires their data. After registration, these third parties can request:

\begin{itemize}
    \item Access to the data of some specific individuals (we can assume, for instance, that they know an 
    individual by his/her social security number or fiscal code in Italy). In this case, TrackMe passes the request to the specific individuals who can accept or refuse it.
    
    \item Access to anonymized data of groups of individuals (for instance, all those living in a certain geographical area, all those of a specific age range, etc.). These requests are handled directly by Data4Help that approves them if it is able to properly anonymize the requested data.\\ 
    For instance, if the third party is asking for data about 10-year-old children living in a certain street in Milano and the number of these children is two, then the third party could be able to derive their identity simply having people monitoring the residents of the street between 8.00 and 9.00 when kids go to school. Then, to avoid this risk and the possibility of a misuse of data, Data4Help will not accept the request. \\ 
    For simplicity, we assume that Data4Help will accept any request for which the number of individuals whose data satisfy the request is higher than 1000.
\end{itemize}
As soon as a request for data is approved, TrackMe makes the previously saved data available to the third party. Also, it allows the third party to subscribe to new data and to receive them as soon as they are produced. \\
TrackMe develops itself two third-party services: AutomatedSOS and Track4Run. \\ \\
\textbf{AutomatedSOS} is a project created by TrackMe and placed on top of Data4Help, which aims to offer a personalized and non-intrusive SOS service to people that are worried about their own safeness and want to have an health inspector always with them. \\ AutomatedSOS, exploiting data provided from Data4Help, monitors the health status of the subscribed customers and, when such parameters are below certain thresholds, sends to the location of the customer an ambulance, guaranteeing a reaction time of less than 5 seconds from the time the parameters are below the threshold. \\ \\
\textbf{Track4Run} is another service offered by TrackMe and conceived for athletes participating in a run, which should allow organizers to define the path for the run, competitors to enroll to the run, and spectators to see on a map the position of all runners during the run. \\
Of course, also in this case, Track4Run will exploit the features offered by Data4Help.
\clearpage

\subsection{Definitions, Acronyms, Abbreviation}

\subsubsection{Definitions}
\begin{itemize}
    \item \textit{Customer:} generic person not registered in any of the services provided by TrackMe. 
    \item \textit{User:} person registered in at least one of the services provided by TrackMe.
    \item \textit{Third Party:} person or company registered in Data4Help that wants to access to the service in order to acquire data.
    \item \textit{Individual Data:} data that belong to a specific user, in particular personal information and data acquired from their devices (i.e. location, health status, etc.).
    \item \textit{Anonymized Data:} aggregated data acquired from a group of users with specific characteristics without keeping track of who they belong to.
    \item \textit{Anonymized Data Subscription:} request that allows third parties to receive new anonymized data as soon as they are produced.
    \item \textit{Ambulance Dispatcher:} the contact center that handles the health care service and the ambulance allocation.
    \item \textit{Google Maps:} web mapping service developed by Google offering satellite imagery, street maps etc.
    \item \textit{Push Notification Service:} web service that offers a fast and secure channel to notify and exchange messages on the web.
    % TODO: rivedere
\end{itemize}

\subsubsection{Acronyms}
\begin{itemize}
    \item \textit{DD:} Design Document
    \item \textit{RASD:} Requirement, Analysis and Specification Document
    \item \textit{GPS:} Global Position System
    \item \textit{API:} Application Programming Interface
    \item \textit{DBMS:} DataBase Management System
    \item \textit{GUI:} Graphical User Interface
    \item \textit{MVC:} Model-View-Controller pattern
    % TODO: rivedere
\end{itemize}

\subsubsection{Abbreviations}
\begin{itemize}
    \item \textit{R.n} Requirement number n
    % TODO: rivedere
\end{itemize}
\clearpage    

% REVISION HISTORY
% REFERENCE DOCUMENTS

\subsection{Document Structure}
The document consists of five sections:
\begin{enumerate}[label*=\bf{\arabic*.}]
    \item \textbf{Introduction:} this section provides a general introduction to the Design Document and aims to establish its purpose and scope, giving some guidelines for the correct understanding of the document.
    \item \textbf{Architectural Design:} this section gives an overview of the project structure. Indeed, it includes diagrams which show components, sub-components and interfaces, underlining the relationship among them. The analysis starts from a high-level view and it goes down deeper and deeper to touch more specific entities, considering also architectural styles and patterns.  
    \item \textbf{User Interface Design:} this section provides an overview on how the user interfaces of the system look like and behave depending on the different inputs.
    \item \textbf{Requirements Traceability:} this section describes how requirements defined in the RASD are mapped to the design elements shown in this document.
    \item \textbf{Implementation, Integration and Test Plan:} this section explains the order in which has been decided to implement the sub-components of the system and in which has been planned to integrate those sub-components, describing the testing phase for the entire structure.
\end{enumerate}

\clearpage

\section{Architectural Design}

\subsection{Overview}

\subsection{High-level Architecture}

\clearpage

\subsection{Component View}
Services:
\begin{enumerate}
    %Data4Help
    \item RegistrationService
    \item LoginService
    \item PermissionRequestService %Riceve le richieste di permessi e le smista in base al tipo ai due servizi scritti sotto
    \item AnonymousPermissionMaker
    \item IndividualPermissionMaker
    \item IndividualPermissionService %gestisce le individual
    \item AnonymousPermissionService %gestisce le  anonime
    \item IndividualDataService %Accede al db e elabora i dati per mandarli in uscita
    \item AnonymousDataService %Accede al db e elabora i dati per mandarli in uscita
    
    %AutomatedSOS
    \item MonitorStatusService
    \item EmergencyService
    
    %Track4Run
    \item RunCreationService %creazione corsa
    \item RunManagementService %Gestione pre corsa(Aggiunta partecipanti)
    \item RunTrackingService %Gestione durante la corsa
    \item RunWatchService %Gestione spettatori durante la corsa
    
\end{enumerate}

\subsection{Deployment View}
\clearpage

\subsection{Runtime View}

\subsection{Component Interfaces}

\subsection{Selected Architectural Styles and Patterns}

\subsection{Other Design Decision}

\section{User Interface Design}

\section{Requirements Traceability}

\begin{enumerate}[label*=\bf{R.\arabic*}]

\item - People can create a user account selecting username, password,
giving personal information (age, address,gender) and allow to
share their anonymized data.

\begin{itemize}
\item RegistrationService
\end{itemize}

\item - It is possible to create a third party account selecting username
and password and giving the company main information.

\begin{itemize}
\item LoginService
\end{itemize}

\item - Data4Help allows third parties to request anonymized data acquired
from a filtered group of users (by age, gender, address, etc.).

\begin{itemize}
\item DataRequestService
\end{itemize}

\item Data4Help collects data from registered users and gives access to
third parties only if the number of individuals whose data satisfy the
request is higher than 1000.

\begin{itemize}
\item DataReceiveService
\item AnonymousPermissionService
\item AnonymousDataService
\end{itemize}

\item - Data4Help allows third parties to join the subscription service for an
indeterminate period and then, in case, unsubscribe from that.

\begin{itemize}
\item DataRequestService
\end{itemize}

\item - Data4Help provides new data, checking each time if groups data satisfy the given constraint (number of individuals not lower than a
thousand).

\begin{itemize}
\item AnonymousPermissionService
\item AnonymousDataService
\end{itemize}

\item - Data4Help keeps track of the associations between third parties subscriptions and the required group data.

\begin{itemize}
\item AnonymousPermissionService
\end{itemize}

\item - Data4Help allows third parties to require specific person’s data.

\begin{itemize}
\item DataRequestService
\end{itemize}

\item - Data4Help forwards requests from third parties to the demanded
users which can accept or refuse to share their own personal data.

\begin{itemize}
\item DataRequestService
\end{itemize}

\item - Data4Help keeps track of the connections between a specific user and
the third parties which can access to his/her data.

\begin{itemize}
\item IndividualPermissionService
\end{itemize}

\item - Data4Help allows third parties to have access to demanded users’
data each time they need them.

\begin{itemize}
\item IndividualPermissionService
\item IndividualDataService
\end{itemize}

\item - Users’ health and position information received by Data4Help are
analyzed and compared with the threshold values.

\begin{itemize}
\item MonitorStatusService
\end{itemize}

\item - In case of emergency (the health status values overcome the threshold) a request for an ambulance is sent to the ambulance dispatcher
in 5 seconds, containing the user’s information.

\begin{itemize}
\item EmergencyService
\end{itemize}

\item - Users select the day, the hour at which the run begins, the starting
point, the ending point and the path for the run.

\begin{itemize}
\item RunCreationService
\end{itemize}

\item - The run event is stored in the system in order to be managed during
its lifecycle.

\begin{itemize}
\item RunCreationService
\end{itemize}

\item - Users can browse among the available runs and see their information.

\begin{itemize}
\item RunManagementService
\end{itemize}

\item - Users can choose a run and register to it as a runners.

\begin{itemize}
\item RunManagementService
\end{itemize}

\item - The system saves the enrolled users as runners and keeps track of the
association with the run event.

\begin{itemize}
\item RunManagementService
\end{itemize}

\item - The system keeps track of all the enrolled runners’ position during
the run.

\begin{itemize}
\item RunTrackingService
\end{itemize}

\item - Users who require to watch a run are saved as spectators to the run
event.

\begin{itemize}
\item RunManagementService
\end{itemize}

\item - Track4Run offers the possibility to see the run live through a map.

\begin{itemize}
\item RunWatchService
\end{itemize}

\end{enumerate}

\section{Implementation, Integration and Test Plan}

\section{Effort Spent}

\section{References}

\end{document}
