\documentclass[a4paper]{article}
\usepackage[T1]{fontenc}
\usepackage[utf8]{inputenc}
\usepackage[italian]{babel}
\usepackage{courier}
\usepackage{float}
\usepackage{subcaption}
\usepackage{enumitem}
\usepackage{graphicx}
\usepackage{rotating}
\usepackage{tikz}
\usepackage{makecell}
\graphicspath{{./images/}}
\usepackage{tabularx}
%\usepackage{geometry} 
%\geometry{a4paper,top=3cm,bottom=3cm,left=3.5cm,right=3.5cm,% heightrounded,bindingoffset=5mm}

\begin{document}

\author{Manuel Trivilino, Davide Salaorni, Luca Terracciano}

\title{\Large \textbf{Data4Help DD}\\
\textbf{Design Document}
    \begin{figure}[h]
        \centering
        \includegraphics[width=270pt]{data4helpblu}
        \label{fig:my_label_1}
    \end{figure}
    \vspace{-2.4cm}
    \begin{figure}[h]
        \centering
        \includegraphics[width=200pt]{AutomatedSOSlogo.png}
        \label{fig:my_label_2}
    \end{figure}
    \vspace{-2cm}
    \begin{figure}[h]
        \centering
        \includegraphics[width=150pt]{Track4Runlogo.png}
        \label{fig:my_label_3}
\end{figure}
}

\maketitle
\newpage

\tableofcontents
\newpage

\section{Introduction}

\subsection{Purpose}
This document aims to give to the developer team more technical details about the implementation of Data4Help, AutomatedSOS and Track4Run identifying the high level architecture, the design patterns used and the plans that will guide the development of the code.

\subsection{Scope}

\clearpage

\subsection{Definitions, Acronyms, Abbreviation}

\subsubsection{Definitions}
\subsubsection{Acronyms}
\subsubsection{Abbreviation}

\clearpage    

\subsection{Document Structure}
\clearpage

\section{Architectural Design}

\subsection{Overview}

\subsection{High-level Architecture}

\clearpage

\subsection{Component View}

Services:\newline

\begin{enumerate}
    %Data4Help
    \item RegistrationAndLoginService
    \item DataReceiveService %Raccoglie i dati in entrata (Specifici in ogni caso)
    \item DataDeliveryService %Consegna i dati alle terze parti, in forma anonima e non
    \item DataRequestService %Riceve le richieste e le smista in base al tipo ai due servizi scritti sotto
    \item IndividualPermissionService %gestisce le individual
    \item AnonymousPermissionService %gestisce le  anonime
    \item IndividualDataService %Accede al db e elabora i dati per mandarli in uscita
    \item AnonymousDataService %Accede al db e elabora i dati per mandarli in uscita
    
    %AutomatedSOS
    \item MonitorStatusService
    \item EmergencyService
    
    %Track4Run
    \item RunCreationService %creazione corsa
    \item RunManagementService %Gestione pre corsa(Aggiunta partecipanti)
    \item RunTrackingService %Gestione durante la corsa
    \item RunWatchService %Gestione spettatori durante la corsa
    
\end{enumerate}

\subsection{Deployment View}
\clearpage

\subsection{Runtime View}

\subsection{Component Interfaces}

\subsection{Selected Architectural Styles and Patterns}

\subsection{Other Design Decision}

\section{User Interface Design}

\section{Requirements Traceability}

\begin{enumerate}[label*=\bf{R.\arabic*}]

\item - People can create a user account selecting username, password,
giving personal information (age, address,gender) and allow to
share their anonymized data.

\begin{itemize}
\item RegistrationAndLoginService
\end{itemize}

\item - It is possible to create a third party account selecting username
and password and giving the company main information.

\begin{itemize}
\item RegistrationAndLoginService
\end{itemize}

\item - Data4Help allows third parties to request anonymized data acquired
from a filtered group of users (by age, gender, address, etc.).

\begin{itemize}
\item DataRequestService
\end{itemize}

\item Data4Help collects data from registered users and gives access to
third parties only if the number of individuals whose data satisfy the
request is higher than 1000.

\begin{itemize}
\item DataReceiveService
\item AnonymousPermissionService
\item AnonymousDataService
\end{itemize}

\item - Data4Help allows third parties to join the subscription service for an
indeterminate period and then, in case, unsubscribe from that.

\begin{itemize}
\item DataRequestService
\end{itemize}

\item - Data4Help provides new data, checking each time if groups data satisfy the given constraint (number of individuals not lower than a
thousand).

\begin{itemize}
\item AnonymousPermissionService
\item AnonymousDataService
\end{itemize}

\item - Data4Help keeps track of the associations between third parties subscriptions and the required group data.

\begin{itemize}
\item AnonymousPermissionService
\end{itemize}

\item - Data4Help allows third parties to require specific person’s data.

\begin{itemize}
\item DataRequestService
\end{itemize}

\item - Data4Help forwards requests from third parties to the demanded
users which can accept or refuse to share their own personal data.

\begin{itemize}
\item DataRequestService
\end{itemize}

\item - Data4Help keeps track of the connections between a specific user and
the third parties which can access to his/her data.

\begin{itemize}
\item IndividualPermissionService
\end{itemize}

\item - Data4Help allows third parties to have access to demanded users’
data each time they need them.

\begin{itemize}
\item IndividualPermissionService
\item IndividualDataService
\end{itemize}

\item - Users’ health and position information received by Data4Help are
analyzed and compared with the threshold values.

\begin{itemize}
\item MonitorStatusService
\end{itemize}

\item - In case of emergency (the health status values overcome the threshold) a request for an ambulance is sent to the ambulance dispatcher
in 5 seconds, containing the user’s information.

\begin{itemize}
\item EmergencyService
\end{itemize}

\item - Users select the day, the hour at which the run begins, the starting
point, the ending point and the path for the run.

\begin{itemize}
\item RunCreationService
\end{itemize}

\item - The run event is stored in the system in order to be managed during
its lifecycle.

\begin{itemize}
\item RunCreationService
\end{itemize}

\item - Users can browse among the available runs and see their information.

\begin{itemize}
\item RunManagementService
\end{itemize}

\item - Users can choose a run and register to it as a runners.

\begin{itemize}
\item RunManagementService
\end{itemize}

\item - The system saves the enrolled users as runners and keeps track of the
association with the run event.

\begin{itemize}
\item RunManagementService
\end{itemize}

\item - The system keeps track of all the enrolled runners’ position during
the run.

\begin{itemize}
\item RunTrackingService
\end{itemize}

\item - Users who require to watch a run are saved as spectators to the run
event.

\begin{itemize}
\item RunManagementService
\end{itemize}

\item - Track4Run offers the possibility to see the run live through a map.

\begin{itemize}
\item RunWatchService
\end{itemize}

\end{enumerate}

\section{Implementation, Integration and Test Plan}

\section{Effort Spent}

\section{References}

\end{document}
